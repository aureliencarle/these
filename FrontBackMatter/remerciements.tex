%*******************************************************
% Acknowledgments
%*******************************************************
\pdfbookmark[1]{Acknowledgments}{acknowledgments}

\begingroup
\let\clearpage\relax
\let\cleardoublepage\relax
\let\cleardoublepage\relax
\chapter*{Remerciements}

Dans un premier temps je voudrais remercier les gens qui n'ont franchement servi à rien mais que je regretterais de ne pas citer. 
Je peux penser à mes amis d'enfance, ceux qui étaient là durant ma soutenance, Marth, Anto, Val et Val, ceux qui sont venus au bar après (sans plus de commentaires) Flo et Tim. 
\newline

Merci à ma famille. Mes parents pour m'avoir créé, mon frère et ma sœur pour avoir permis à ma mère d'avoir son quota pour la retraite anticipée.
\newline

Je vais remercier les gens qui même extérieures à ma thèse ont eu une relation privilégiée avec moi (enfin privilégiée ...). Clément pour avoir eu accès aux secrets les plus sombres de l'IP2I, Karim pour m'avoir permis d'accéder à mon bureau. Que serait la vie d'un thésard à l'IP2I sans Bruno ? C'est facile c't'une légende, les gens de sa partie l'appellent le Dabe et enlèvent leur chapeau rien qu'en entendant son blaze. Une épée, quoi ! Plus en profondeur chez les figurants, on trouvera Sonia pour la cérémonie des diplômes. Jean-Michel, responsable de l'UE de phy pour les bio, fut (enfin est) aussi quelqu'un qui m'a profondément touché.
\newline

Il y a ensuite l'équipe CMS dans son ensemble. Ça peut aller de Colin à Suzanne en passant par Viola. Même si cette dernière a eu l'intelligence de quitter le navire alors qu'il sombrait peu à peu dans les abysses des modèles à 800 paramètres libres. 
\newline

Mine de rien ma petite aventure à l'APPN mérite de petits remerciements à la volée. Merci à Maryvonne, Yannick, Sylvie, Didier, Djamila, Franck et tous les autres.
\newline

Avant de passer au gratin... (de patates : les thésards quoi) je tiens à remercier très chaleureusement Sylvie (la même que l'APPN) et Stéphanie pour m'avoir de manière différente épaulé et conseillé pendant cette thèse. Une aide plus psychologique pour Sylvie, plus scientifique pour Stéphanie. Elles furent néanmoins toutes deux centrales à la maturation du thésard que j'étais et ce tel ... la licence poétique de ce remerciement sera laissée comme exercice au lecteur. 
\newline

Ayé !! C'est le moment de remercier les autres kékosses (et docteurs depuis pour certains). Je pourrais en écrire des romans sur Gaël et notre similaire vie de doctorant, des romans sur les étranges ressemblances entre Hugues et le Marquis de Sade, des romans (l'anaphore est finie, promis) sur tous les thésards de 3e année quand j'étais en première Robin, Brice et Albert. Mais évidemment les plus importants furent les thésards de ma promo. Corentin avec qui j'ai partagé le représentanat des doctorants, Shahram avec qui j'ai partagé mon dégoût du PSG, Amélie avec qui j'ai partagé Shahram, Jean-François avec qui j'ai partagé de la mauvaise foi avec un accent rustique (le type agricole), Lysandra avec qui j'ai partagé... le reste appartient à l'Histoire, Antoine avec qui j'ai partagé des moments d'ultime génance, Lucas avec qui j'ai partagé mon bureau, Nicolas et Chloé du GANIL avec qui je n'ai pas assez partagé, Quentin avec qui j'ai partagé la mère de Grégoire et Grégoire avec qui j'ai partagé ma dignité.
Merci à Yannick et Guilliam pour avoir fait le déplacement à la soutenance et pour exister tout simplement. Grégory.
 C'est aussi ici que je vais remercier Cécile pour m'avoir soutenu (une façon politiquement correcte de dire supporter). 
\newline

Maintenant les gens utiles.
Je veux remercier très chaleureusement Sabine Crépé-Renaudin et Christophe Delaere d'avoir accepté de relire ma thèse, et d'avoir participé à ma soutenance, un des moments les plus importants de ma vie. Merci du fond du cœur à Christophe Le Poncin-Lafitte pour avoir partagé avec moi des souvenirs merveilleux mais illégaux dans la juridiction de l'Indiana. Et merci du fond du cœur à Corinne Augier qui m'a permis de répondre à une question existentielle. Oui Dieu existe et c'est une femme. 
\newline

Avant d'exprimer ma reconnaissance à Nicolas et Stéphane, en vrai, je devrais encore une fois remercier Grégoire parce que c'est mon meilleur ami ... mais j'écris ces remerciements avec, à coté de moi, la peluche Koopa Troopa qui me fais conclure que je vais m'abstenir d'un second remerciement. <3
\newline

Enfin donc merci à Nicolas Chanon et Stéphane Perriès, mes deux directeurs de thèse. Ils m'ont permis d'avoir accès et de réussir un sujet en or qui m'a fait faire de la physique partant de la phénoménologie à l'expérimental. Un travail de physicien complet. Je les remercie tellement profondément que ce paragraphe ne comporte ni blagues nulles ni références capillotractés à Audiard. Merci tout simplement. <3
\newline


\endgroup
