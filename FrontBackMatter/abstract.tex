%*******************************************************
% Abstract
%*******************************************************
%\renewcommand{\abstractname}{Abstract}
\pdfbookmark[1]{Résumé}{resume}
\begingroup
\let\clearpage\relax
\let\cleardoublepage\relax
\let\cleardoublepage\relax

\chapter*{Résumé}

La symétrie de Lorentz (symétrie d'espace-temps de la relativité restreinte) est un pilier de toute la physique moderne. Elle peut être brisée dans certaines théories au-delà du modèle standard, donnant lieu à une signature dépendante du temps qui peut être recherchée au LHC.

Cette thèse présente deux études de la violation de la symétrie de Lorentz dans le secteur du quark top. La première, phénoménologique, présente les perspectives de recherches de cette brisure de symétrie avec la production de paire de quark top-antitop au LHC et dans d'autres scenarii de collisionneurs. Cette recherche abouti avec l'obtention de valeur précisions sur la mesure de la violation. La second partie sera destinée à la présentation de l'analyse réalisées avec les données de CMS pour le Run II du LHC. Cette recherche confirme les résultats de l'étude phénoménologique tout en offrant des perspectives recherches futurs. 



\vfill

\pdfbookmark[1]{Abstract}{abstract}
\chapter*{Abstract}

The Lorentz symmetry (the space-time symmetry of special relativity) is a pillar of all modern physics. It can be broken in some theories beyond the standard model, giving rise to a time-dependent signature that can be searched for at the LHC.

This thesis presents two studies of Lorentz symmetry violation in the top quark sector. The first one, phenomenological, presents the perspectives of the search for this symmetry breaking with the production of top-antitop quark pairs at the LHC and in other collider scenarios. This research has been successful in obtaining precise measurements of the violation. The second part will be dedicated to the presentation of the analysis performed with CMS data for the LHC Run II. This research confirms the results of the phenomenological study while offering perspectives for future research. 


\endgroup			

\vfill
