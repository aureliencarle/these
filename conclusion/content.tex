\chapter*{Conclusion}
\addcontentsline{toc}{chapter}{Conclusion}

Si le Modèle Standard fait preuve d'une admirable robustesse face à l'épreuve du test expérimental, avec la découverte en 2012 du boson de Higgs, il n'est néanmoins pas exempt de problèmes théoriques. Un des problèmes majeurs est l'impossibilité du Modèle Standard à faire une description (au moins naïve) de la gravitation. Pour pallier ce problème de "gravitation quantique", plusieurs approches théoriques sont proposées par la communauté scientifique, notamment la théorie des cordes, la gravitation quantique à boucle ou encore la géométrie non-commutative. Ces approches ont en commun la possibilité d'une violation de la symétrie de Lorentz. Cette violation est donc un point d'étude pertinent pour la recherche de gravitation quantique. 
Une théorie effective a été proposée par Alan Kosteleck\`y et Dan Colladay, pour permettre l'étude de la violation de la symétrie de Lorentz en physique des particules : The Standard Model Extension (SME). Cette théorie effective met en avant des coefficients de Wilson pouvant être interprétés comme une valeur non nulle d'un champs non-scalaires dans le vide. Si un phénomène à basse énergie de violation de la symétrie de Lorentz suit la même tendance qu'un couplage de Yukawa, alors le quark \Ptop est le secteur dans lequel un signal peut être attendu.
\newline

Le travail de recherche présenté dans cette thèse s'articule autour de deux axes principaux. Premièrement une étude phénoménologique cherchant à établir la sensibilité potentielle d'une recherche de violation de Lorentz avec une expérience au LHC pour le Run II avec une extrapolation pour des scenarii de futurs collisionneurs. Dans un second temps l'analyse des données collectées par la collaboration CMS a été effectuée. Ceci a établi la concordance de l'étude précédemment réalisée. 
\newline


Une étude de faisabilité du SME pour le processus $\ttbar \rightarrow \Pbottom \Pleptonplus \Pnulepton +  \APbottom \Pleptonminus \APnulepton$ a permis de prévoir une amélioration, par rapport à l'étude menée par D$\emptyset$, pour la mesure des coefficients de Wilson du SME de l'ordre de \num{e2} à \num{e3}. La mesure de D$\emptyset$, unique mesure de ces coefficients dans le secteur du top, établit une absence de violation avec une incertitude de \num{10}{\%}. Notre prévision pour des scenarii de collisionneurs hadron-hadron vont jusqu'à des augmentations de facteurs   \num{e5} à \num{e6} sur la précision sur les coefficients. 

Suite à l'obtention de ces résultats, nous avons entrepris au sein de la collaboration CMS une analyse pour mesurer les coefficients de Wilson. Une étude utilisant une méthode du maximum de vraisemblance a été utilisée pour la mesure dans le cadre d'un test Asimov. Cette analyse s'est concrétisée par une validation des résultats phénoménologiques et même une augmentation de la précision d’un facteur \num{2} à \num{3}. L'étape suivante sera, après validation de la collaboration CMS, la mesure sur les données enregistrées par le détecteur. 
\newline

Pour faire suite à ce travail de recherche, plusieurs pistes peuvent être envisagées. Un travail plus fin sur les incertitudes systématiques dépendantes du temps pourrait être une première étape. Une analyse sur les données 2018 pourrait être une suite envisageable. Une idée pourrait être également de travailler à l'amélioration de la discrimination signal sur bruit de fond avec l'utilisation d'autres variables discriminantes.
\newline

De manière plus éloignée, le travail phénoménologique du chapitre \ref{chap:chap2} a servi de base à une étude de faisabilité sur la mesure de la violation CPT (symétrie reliée à la symétrie de Lorentz) avec des processus de quark top solitaire. De la même manière, l'analyse faite au chapitre \ref{chap:chap5} pourrait être le point de départ pour l'investigation de ce secteur.