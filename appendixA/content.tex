\chapter{Annexe : Outils mathématiques} \label{chap:appA}
% PRL Mtt

\begin{fmffile}{appendixA}

\section{Structure algébrique}

\subsection{Groupe}\label{A:groupe}

En mathématique un groupe est un couple constitué d'un ensemble $G$ et d'une opération $\ast$. On note $\left( G, \ast\right)$ le groupe doté d'un ensemble $G$ muni d'une loi de composition $\ast$. 
Un groupe doit respecter quatre caractéristiques : 
\begin{itemize}[label=$\triangleright$]
    \item La loi de composition doit être interne : \newline
    $\forall a,b \in G, \quad a\ast b \in G$.
    \item La loi de composition doit être associative : \newline
    $ (a \ast b) \ast c = a \ast (b \ast c)$.
    \item L'existence d'un élément neutre $e\in G$ : \newline
    $\forall a,b \in G, \quad a\ast e = e \ast a = a$.
    \item L'existence d'un élément symétrique dans $G$ : \newline
    $\forall a \in G, \, \exists s \in G,  \quad a\ast s = s \ast a = e $. On note le symétrique de $a$, $a^{-1}$.
\end{itemize}
Éventuellement, un groupe peut être muni d'une dernière contrainte. Si l'opération est commutative $\left( \forall a,b \in G\setminus\{e\} , \,  a*b = b*a \right)$, le groupe est dit abélien.


\subsection{Algèbre}\label{A:algebre}
Pour construire une algèbre il faut batir une structure intermédiaire. 
Soit un groupe auquel on ajoute un ensemble $\mathbb{K}$ muni d'une opération addition $+$  et de multiplication $\cdot$ que l'on lie par les relations suivantes : 
\begin{equation}
    \lambda \cdot (u + v) = \lambda \cdot u + \lambda \cdot v  \qquad \forall u,v \in G, \, \forall \lambda \in \mathbb{K}
\end{equation}
\begin{equation}
    (\lambda + \mu) \cdot u = \lambda \cdot u + \mu \cdot u  \qquad \forall u \in G, \, \forall \lambda,\mu \in \mathbb{K}
\end{equation}
\begin{equation}
    (\lambda \mu) \cdot u = \lambda \cdot (\mu \cdot u)  \qquad \forall u \in G, \, \forall \lambda,\mu \in \mathbb{K}
\end{equation}
Cette structure est un espace vectoriel si $(\mathbb{K},+, \cdot)$ est un corps, un module si $(\mathbb{K},+, \cdot)$ est un anneau.

Une algèbre noté $\left( G, \mathbb{K}, +, \cdot, \star \right)$ est un espace vectoriel ou module auquel on ajoute une opération bilinéaire : 
\begin{align*}
\star : G \times G &\longrightarrow G \\
     g_1, g_2 &\longmapsto g_3 = g_1 \star g_2
\end{align*}

\textbf{Une algèbre de Lie} est une algèbre $\left( G, \mathbb{K}, +, \cdot, [\,] \right)$ dont l'opération bilinéaire, nommée "crochet de Lie", a pour propriétés : 
\begin{itemize}[label=$\triangleright$]
    \item $[x,y] = - [y,x]  $
    \item $[x,[y,z]] + [y,[z,x]] + [z,[x,y]] = 0$
\end{itemize}
Un espace vectoriel $\mathbb{R}^3$ munie du produit vectoriel est un exemple d'algèbre de Lie. 

\section{Groupe et algèbre de Lie}\label{A:Lie}
Un groupe de Lie peut être intégralement construit à partir d'un ensemble réduit d'objets nommées générateurs et répondant à :
\begin{equation} 
 g = e^{\alpha_k T^k}
\end{equation}  
avec $T^k$ les générateurs du groupe et $\alpha_k$ des paramètres du groupe. 

\subsubsection*{Exemple simple $SO(2)$}
Un exemple très simple est le groupe des matrices de rotations dans l'espace $\mathbb{R}^2$ spéciales orthogonales $\mathrm{SO}(2)$ dont le générateur $T$ est donné par:
\begin{equation}
    T = 
    \begin{pmatrix}
        0 & -1 \\
        1 & 0
    \end{pmatrix}
\end{equation}
Pour un paramètre $\theta$ représentant l'angle de rotation, $R \in \mathrm{SO}(2)$ :  
\begin{align*}
    R &= e^{\theta T} = \sum_{n=0}^{+\infty} \frac{\theta^n}{n!} 
    \begin{pmatrix}
        0 & -1 \\
        1 & 0
    \end{pmatrix}^n \\
    &= \begin{pmatrix}
            1 & 0 \\
            0 & 1
        \end{pmatrix}
      +
      \begin{pmatrix}
              0 & -\theta \\
              \theta & 0
          \end{pmatrix}
      +\frac{1}{2}
      \begin{pmatrix}
               -\theta^2 & 0 \\
              0 &  -\theta^2
      \end{pmatrix} + ... \\
    &= \begin{pmatrix}
            \cos(\theta) & -\sin(\theta) \\
            \sin(\theta) & \cos(\theta)
        \end{pmatrix}
\end{align*}


\subsection{Algèbre et crochet de Lie}
Les générateurs sont éléments d'une algèbre, l'algèbre de Lie. Une algèbre de Lie peut être reconnue par la valeurs du crochet de Lie de ses générateurs. 
\begin{table}[H]\label{A:groupedesmetrie}
\begin{center}
    \begin{tabular}{c|cc}
    \noalign{\smallskip}\hline\noalign{\smallskip}
    Groupe & Crochet de Lie & Générateur  \\
    de Lie &   de l'algèbre& du groupe\\
    \noalign{\smallskip}
    \hline \hline
    \noalign{\smallskip}
     $\mathrm{SU}(2)$ &  $[\frac{\sigma_a}{2}, \frac{\sigma_b}{2}] = i\epsilon_{abc}\frac{\sigma_c}{2}$ & $\{i\sigma_a\}$ matrices de Pauli\\
     $\mathrm{SU}(3)$ &  $[\frac{\lambda_a}{2},\frac{\lambda_b}{2}] = if_{abc}\frac{\lambda_c}{2}$ & $\{i\lambda_a\}$ matrices de Gell-Man\\
     \noalign{\smallskip}\hline\noalign{\smallskip}
     $\mathrm{SO}(1,3)$ & $[J_a, J_b] = i \epsilon_{abc}J_c  $ & $\{J_a\}$ Générateurs de rotation\\
     & $[K_a,K_b] = -i \epsilon_{abc}J_c $ &   $\{K_a\}$ Générateurs de boosts\\ 
     &$[J_a,K_b] = i \epsilon_{abc}K_c $ & \\ 
    \noalign{\smallskip}\hline\noalign{\smallskip}
    $\mathrm{SU}(2)\otimes\mathrm{SU}(2)$ & $[J^+_a, J^+_b] = \frac{i}{2} \epsilon_{abc}J^+_c $ & $\{J^+_a\}$ Générateurs de $SU(2)$\\
     & $[J^-_a, J^-_b] = \frac{i}{2} \epsilon_{abc}J^-_c $ & $\{J^-_a\}$ Générateurs de $SU(2)$\\
    &$[J^+_a, J^-_b] = 0$& \\
    \noalign{\smallskip}\hline\noalign{\smallskip}
    \end{tabular}
\end{center}
\end{table}


\section{Théorie des représentations}\label{A:représentation}

\subsection{Définition}
    
La représentation d'un groupe $\left( G, \ast\right)$ ou d'une algèbre de Lie $\left( G, \mathbb{K}, +, \cdot, [\,] \right)$  est action sur un espace vectoriel V défini par l'application :
\begin{align*}
\Phi : G \times V &\longrightarrow V \\
     v &\longmapsto \Phi(g,v)
\end{align*}
Avec la notation $ \rho(g)$ pour $\Phi(g,v)$, la représentation de $g$ doit suivre la relation : 
\begin{equation}\label{representation}
 \rho(g_1) \cdot  \rho(g_2) = \rho(g_1 g_2) 
\end{equation}

Dans les cas physiques, la représentation est très souvent une matrice agissant sur des espaces vectoriels $\mathbb{R}^n$ ou $\mathbb{C}^n$.

\subsection{La représentation triviale}

On peut remarquer que $\forall g \in G, \quad \rho(g) = \mathbbm{1}$ obéis à la règle \eqref{representation}. Dans un tel cas, tout élément de $G$ est représenté par la matrice identité et on nomme cette représentation la représentation triviale. Alors l'espace vectoriel $V$ est le corps $\mathbb{K}$, autrement dit un espace scalaire.

\subsection{La représentation fondamentale}

La représentation fondamentale est la représentation irréductible (qui ne peut être décomposée en représentations plus petites). Dans la plupart des cas il s'agit des matrices qui définissent le groupe. 


\subsection{Exemples de groupes présent en physique}\label{A:groupedesymetrie}

\begin{table}[H]
\begin{center}
    \begin{tabular}{c|cc}
    \noalign{\smallskip}\hline\noalign{\smallskip}
    & Représentation & Représentation  \\
    Groupe &  triviale & fondamentale \\
    \noalign{\smallskip}
    \hline \hline
    \noalign{\smallskip}
    $U(1)$ & 1 & $\left\{ e^{i\theta} :\, \theta \in \mathbb{R}\right\}$ \\
    $\mathrm{SO}(2)$ &  $\mathbbm{1}$ & $\left\{  M = 
    \begin{pmatrix}
        a & -b \\ b & a
    \end{pmatrix} :\, a,b\in \mathbb{R}, \, a^2 + b^2 =1 
     \right\}$ \\
     $\mathrm{SU}(2)$ &  $\mathbbm{1}$ & $\left\{  M = 
     \begin{pmatrix}
             \alpha & -\overline{\beta} \\ \beta & \overline{\alpha} 
         \end{pmatrix} :\, \alpha,\beta\in \mathbb{C}, \, |\alpha|^2 + |\beta|^2 =1 
      \right\}$ \\
    \noalign{\smallskip}\hline\noalign{\smallskip}
    \end{tabular}
\end{center}
\end{table}




\end{fmffile}
